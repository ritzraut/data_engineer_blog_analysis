\documentclass{article}
\usepackage[a4paper, margin=2cm]{geometry}
\usepackage{fontspec}
    \setmainfont{Lato}
\setlength\parindent{0pt}
\usepackage[most]{tcolorbox}
\definecolor{ee-blue}{RGB}{142,209,252}
\newtcolorbox{formula}{colback=ee-blue,grow to right by=-10mm,grow to left by=-10mm,
boxrule=0pt,boxsep=0pt,breakable}
\pagenumbering{gobble}


\date{}
\title{Calculating the Outliers}
\author{Equal Experts Data Engineer Challenge}
\begin{document}

  \maketitle
  \section*{Calculation}
    A week is classified as an outlier when the total votes for the week deviate from the average votes per week for the complete dataset by more than 20\%. For the avoidance of doubt, \textit{please use the following formula}:

    \begin{formula}
    Say the mean votes is given by $\bar{x}$ and this specific week's votes is given by $x_i$.\\
    We want to know when $x_i$ differs from $\bar{x}$ by more than 20\%. \\
    When this is true, then the ratio $\frac{x_i}{\bar{x}}$ must be further from $1$ by more than $0.2$, i.e.:

    $$
    \big|1 - \frac{x_i}{\bar{x}}\big| > 0.2
    $$
    \end{formula}

    We want this outlier calculation's output to be stored in the view called \texttt{outlier\_weeks}.

    The data should be sorted in the view by year and week number, with the earliest week first.

  \section*{Test Data}

  Given the following test data:
  \begin{verbatim}
{"Id":"1","PostId":"1","VoteTypeId":"2","CreationDate":"2022-01-02T00:00:00.000"}
{"Id":"2","PostId":"1","VoteTypeId":"2","CreationDate":"2022-01-09T00:00:00.000"}
{"Id":"4","PostId":"1","VoteTypeId":"2","CreationDate":"2022-01-09T00:00:00.000"}
{"Id":"5","PostId":"1","VoteTypeId":"2","CreationDate":"2022-01-09T00:00:00.000"}
{"Id":"6","PostId":"5","VoteTypeId":"3","CreationDate":"2022-01-16T00:00:00.000"}
{"Id":"7","PostId":"3","VoteTypeId":"2","CreationDate":"2022-01-16T00:00:00.000"}
{"Id":"8","PostId":"4","VoteTypeId":"2","CreationDate":"2022-01-16T00:00:00.000"}
{"Id":"9","PostId":"2","VoteTypeId":"2","CreationDate":"2022-01-23T00:00:00.000"}
{"Id":"10","PostId":"2","VoteTypeId":"2","CreationDate":"2022-01-23T00:00:00.000"}
{"Id":"11","PostId":"1","VoteTypeId":"2","CreationDate":"2022-01-30T00:00:00.000"}
{"Id":"12","PostId":"5","VoteTypeId":"2","CreationDate":"2022-01-30T00:00:00.000"}
{"Id":"13","PostId":"8","VoteTypeId":"2","CreationDate":"2022-02-06T00:00:00.000"}
{"Id":"14","PostId":"13","VoteTypeId":"3","CreationDate":"2022-02-13T00:00:00.000"}
{"Id":"15","PostId":"13","VoteTypeId":"3","CreationDate":"2022-02-20T00:00:00.000"}
{"Id":"16","PostId":"11","VoteTypeId":"2","CreationDate":"2022-02-20T00:00:00.000"}
{"Id":"17","PostId":"3","VoteTypeId":"3","CreationDate":"2022-02-27T00:00:00.000"}
  \end{verbatim}

  You should have the following in your \texttt{outlier\_weeks} view: \\
  \\
  \begin{tabular}{|r|r|r|}
  \hline
  Year & WeekNumber & VoteCount \\
  \hline
    2022 &  0 &  1 \\
    2022 &  1 &  3 \\
    2022 &  2 &  3 \\
    2022 &  5 &  1 \\
    2022 &  6 &  1 \\
    2022 &  8 &  1 \\
  \hline
  \end{tabular}
    \\
    \\
  \textbf{Note that we strongly encourage you to use this data as a test case to ensure that you have the correct calculation!}
\end{document}
